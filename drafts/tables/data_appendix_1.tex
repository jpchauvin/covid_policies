%\usepackage{booktabs}
%\usepackage{multirow}
%\usepackage{rotating}
\renewcommand{\arraystretch}{1.3}
    \begin{tabular}{lp{47.5em}}
    \textbf{Definitions} & \multicolumn{1}{l}{\textbf{Description / comments}} \\
    \midrule
    Working age population & Individuals between 15 and 64 years old in the period of reference. \\
    Formally employed & Individual that worked over the period of reference with a signed work card , or was an employer. \\
    Informally employed & Individual that worked over the period of reference without a signed work card, or was self-employed.  \\
    Employed & Individual either formally or informally employed. \\
    Non-employed & Working age individual declared as non employed. \\
    Migrant & Individual that declares that its time of residence in their current municipality is less or equal to 5 years.  \\
    High skill & Individuals that completed at least high-school-equivalent education (2do grau, colegial o medio 2do ciclo). \\
    Wage  & Monthly labor income in main occupation in the reference period. \\
    Rent* & Monthly value of housing rent. \\
    Industry of employment & Four major industries based on CNAE - Domiciliar definition \\
    Catchment area** & Set of rural municipalities with a positive rate of emigration to a given city. The rate of  emigration uses individuals who migrated 5 to 10 years before the reference period. \\
    Arranjos populacional & Grouping of two or more municipalities where there is a strong population integration due to commuting to work or study, or due to contiguity between the spots main urbanized areas (\citealp{IBGE2016}.) We use a time-consistent definition joining the arranjos that share a common municipality for the period 1970-2010 . \\
    Urban  & Individual living inside an arranjo populacional and identified as living in an urban area in the census of the reference period. \\
    Rural & Individual living outside an arranjo populacional and identified as living in a rural area in the census of the reference period. \\
    \bottomrule
    \end{tabular}%
