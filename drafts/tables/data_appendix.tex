%\usepackage{booktabs}
%\usepackage{multirow}
%\usepackage{rotating}


\begin{sidewaystable}[htbp]
  \caption{Definitions}\\


    \begin{tabular}{lp{47.5em}}
    \textbf{Definitions} & \multicolumn{1}{l}{\textbf{Description / comments}} \\
    \midrule
    Working age population & Individuals between 15 and 64 years old in the period of reference. \\
    Formally employed & Individual that worked over the period of reference with a signed work card , or was an employer. \\
    Informally employed & Individual that worked over the period of reference without a signed work card, or was self-employed.  \\
    Employed & Individual either formally or informally employed. \\
    Non-employed & Working age individual declared as non employed. \\
    Migrant & Individual that declares that its time of residence in their current municipality is less or equal to 5 years.  \\
    High skill & Individuals that completed at least high-school-equivalent education (2do grau, colegial o medio 2do ciclo). \\
    Wage  & Monthly labor income in main occupation in the reference period. \\
    Rent* & Monthly value of housing rent. \\
    Industry of employment & Four major industries based on CNAE - Domiciliar definition \\
    Catchment area** & Set of rural municipalities with a positive rate of emigration to a given city. The rate of  emigration uses individuals who migrated 5 to 10 years before the reference period. \\
    Arranjos populacional & Grouping of two or more municipalities where there is a strong population integration due to commuting to work or study, or due to contiguity between the spots main urbanized areas \citet{IBGE2016}. We use a time-consistent definition joining the arranjos that share a common municipality for the period 1970-2010 . \\
    Urban  & Individual living inside an arranjo populacional and identified as living in an urban area in the census of the reference period. \\
    Rural & Individual living outside an arranjo populacional and identified as living in a rural area in the census of the reference period. \\
    \bottomrule
    \end{tabular}%
  \label{tab:addlabel}%
	 \begin{tablenotes}
			\footnotesize \textbf{Notes.} * Housing rents are not available for the 2000 census. We discarded the upper 1\%  extreme values to compute the average values of this variable.

			** The 2000 census does not allow to identify the municipality of the previous residence for individuals who migrated before 1995. We use instead, the share of migrants between 1991-1986 (from the 1991 census) as weights for this year.

			\end{tablenotes}
\end{sidewaystable}%

\begin{sidewaystable}[htbp]

  \caption{Variables related to the HT prediction error}
    \begin{tabular}{p{14.445em}rp{37.89em}}
    \multicolumn{1}{l}{\textbf{Variable}} & \multicolumn{1}{l}{\textbf{Samples}} & \multicolumn{1}{l}{\textbf{Description / comments}} \\
    \midrule
    \multicolumn{1}{r}{} &       & \multicolumn{1}{r}{} \\
    Average rural wage* & \multicolumn{1}{p{7.055em}}{1991, 2000, 2010} & Average log-wage of invididuals identified as living in rural areas in the catchment area of a city.  \\
    Minimum wage & \multicolumn{1}{p{7.055em}}{1991, 2000, 2010} & National minimum wage published by the Ministry of Labor and Employment in the reference period, in 2010 reais. \\
    Average urban wage** & \multicolumn{1}{p{7.055em}}{1991, 2000, 2010} & Average log-wage of individuals identified as living in urban areas of a city.  \\
    Average formal urban wage & \multicolumn{1}{p{7.055em}}{1991, 2000, 2010} & Average log-wage of individuals identified as formally employed living in urban areas of a city.  \\
    Average informal urban wage & \multicolumn{1}{p{7.055em}}{1991, 2000, 2010} & Average log-wage of individuals identified as informally employed living in urban areas of a city.  \\
    Rural-minimum wage ratio & \multicolumn{1}{p{7.055em}}{1991, 2000, 2010} & Average across cities of the ratio between the average rural wage and minimum wage. \\
    Rural-urban wage ratio & \multicolumn{1}{p{7.055em}}{1991, 2000, 2010} & Average across cities of the ratio between the average rural wage and average urban wage. \\
    Rural-formal urban wage ratio & \multicolumn{1}{p{7.055em}}{1991, 2000, 2010} & Average across cities of the ratio between the average rural wage and average urban formal wage. \\
    Rural-informal urban wage ratio & \multicolumn{1}{p{7.055em}}{1991, 2000, 2010} & Average across cities of the ratio between the average rural wage and average urban informal wage. \\
    Urban housing rent & \multicolumn{1}{p{7.055em}}{1991, 2010} & Average log-rent of households identified as living in urban areas of a city.  \\
    Rural housing rent* & \multicolumn{1}{p{7.055em}}{1991, 2010} & Average log-rent of households identified as living in rural areas in the catchment area of a city.  \\
    Urban non-employed / employed & \multicolumn{1}{p{7.055em}}{1991, 2000, 2010} & Average across cities of the ratio between the number of the ratio between non-employed  and employed individuals. \\
    Informality rate & \multicolumn{1}{p{7.055em}}{1991, 2000, 2010} & Share of informally employed in a city. \\
    \multicolumn{1}{r}{} &       & \multicolumn{1}{r}{} \\
    Urban population prediction error (minimum wage /formal/informal/two urban sectors)*** & \multicolumn{1}{p{7.055em}}{1991, 2000, 2010} & Average across cities of the value of the error defined in equation 4. For each city, the error is computed using as inputs the ratio between employed and non-employed working age-urban population, and the ratio between the average rural wage and the average minimum/urban formal/urban informal/urban wage.  \\
    Urban unemployment prediction error (minimum wage /formal/informal/two urban sectors)*** & \multicolumn{1}{p{7.055em}}{1991, 2000, 2010} &   Average across cities of the value of the error defined in equation 5. For each city, the error is computed using as inputs the ratio between non-employed and employed working age-urban population and the ratio between the average minimum/urban formal/urban informal/urban wage and the average rural wage.  \\
    Urban population prediction error with housing market*** & \multicolumn{1}{p{7.055em}}{1991,  2010} &   Average across cities of the value of the error defined in equation 9. For each city, the error is computed using as inputs the ratio between employed and non-employed working age-urban population, the ratio between the average rural wage and the average urban wage, and the ratio between rural and urban housing rent to the power of 0.33.  \\
    Urban unemployment prediction error with housing market*** & \multicolumn{1}{p{7.055em}}{1991, 2010} &   Average across cities of the value of the error defined in equation 10. For each city, the error is computed using as inputs the ratio between non-employed and employed working age-urban population, the ratio between the average urban wage and the average rural wage, and the ratio between rural and urban housing rent to the power of 0.33.    \\
    Percent of cities within 0.1 &       & Share of errors lying in the interval -0.1  to  0.1  for cities in a reference period. \\


    \bottomrule

    \end{tabular}%
  \label{tab:addlabel}%
 \begin{tablenotes}
			\footnotesize \textbf{Notes.}*  For rural areas, we compute the average value for a location; subsequently, we compute the weighted average of the rural areas within the catchment area of a city.  We used the share of emigrants from each rural municipality to an urban area as weights.  Migration shares represent the total number of individuals who migrated from a rural area to an urban area between 10 and 5 years previous to the census year over the total number of immigrants of the urban area.

 ** It is computed separately for formal and informal workers within a city. Finally, we compute a weighted average of both average values using the share of formality/informality in a given city as weights.

  *** To avoid extreme values, errors are winzorized at 1\% and 99\%.


			\end{tablenotes}

\end{sidewaystable}%

\begin{sidewaystable}[htbp]


  \caption{Correlates of the HT prediction error}
    \begin{tabular}{p{14.445em}rp{37.89em}}
    \multicolumn{1}{l}{\textbf{Variable}} & \multicolumn{1}{l}{\textbf{Samples}} & \multicolumn{1}{l}{\textbf{Description / comments}} \\
    \midrule
    \multicolumn{1}{r}{} &       & \multicolumn{1}{r}{} \\

  Distance* &       & Average distance between the centroid of the most populated municipality within an urban area and the centroid of each rural municipality in the catchment area.  \\
    Population density* &       & Ratio between the working-age population in a municipality and its area in square kilometers. \\
    Rural areas in the catchment area &       & Number of rural municipalities with positive emigration to a city in a reference period. \\
    Population* &       & Working-age population of a municipality. \\
    Share of HS educated* &       & Share of high skilled individuals in the working-age population a of a municipality with education information. \\
    Share of population aged 15-39* &       & Share of individuals aged 15-39 from the working-age population. \\
    Agriculture/Manufacture employment share* &       & Share of employed individuals working in manufacturing/agriculture in the reference period. \\

    \bottomrule

    \end{tabular}%
  \label{tab:addlabel}%
 \begin{tablenotes}
			\footnotesize \textbf{Notes.}*  For rural areas, we compute the average value for a location; subsequently, we compute the weighted average of the rural areas within the catchment area of a city.  We used the share of emigrants from each rural municipality to an urban area as weights.  Migration shares represent the total number of individuals who migrated from a rural area to an urban area between 10 and 5 years previous to the census year over the total number of immigrants of the urban area.


			\end{tablenotes}

\end{sidewaystable}%
